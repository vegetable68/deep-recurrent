%%%%%%%%%%%%%%%%%%%%%%%%%%%%%%%%%%%%%%%%%
% Compact Laboratory Book
% LaTeX Template
% Version 1.0 (4/6/12)
%
% This template has been downloaded from:
% http://www.LaTeXTemplates.com
%
% Original author:
% Joan Queralt Gil (http://phobos.xtec.cat/jqueralt) using the labbook class by
% Frank Kuster (http://www.ctan.org/tex-archive/macros/latex/contrib/labbook/)
%
% License:
% CC BY-NC-SA 3.0 (http://creativecommons.org/licenses/by-nc-sa/3.0/)
%
% Important note:
% This template requires the labbook.cls file to be in the same directory as the
% .tex file. The labbook.cls file provides the necessary structure to create the
% lab book.
%
% The \lipsum[#] commands throughout this template generate dummy text
% to fill the template out. These commands should all be removed when 
% writing lab book content.
%
% HOW TO USE THIS TEMPLATE 
% Each day in the lab consists of three main things:
%
% 1. LABDAY: The first thing to put is the \labday{} command with a date in 
% curly brackets, this will make a new section showing that you are working
% on a new day.
%
% 2. EXPERIMENT/SUBEXPERIMENT: Next you need to specify what 
% experiment(s) and subexperiment(s) you are working on with a 
% \experiment{} and \subexperiment{} commands with the experiment 
% shorthand in the curly brackets. The experiment shorthand is defined in the 
% 'DEFINITION OF EXPERIMENTS' section below, this means you can 
% say \experiment{pcr} and the actual text written to the PDF will be what 
% you set the 'pcr' experiment to be. If the experiment is a one off, you can 
% just write it in the bracket without creating a shorthand. Note: if you don't 
% want to have an experiment, just leave this out and it won't be printed.
%
% 3. CONTENT: Following the experiment is the content, i.e. what progress 
% you made on the experiment that day.
%
%%%%%%%%%%%%%%%%%%%%%%%%%%%%%%%%%%%%%%%%%

%----------------------------------------------------------------------------------------
%	PACKAGES AND OTHER DOCUMENT CONFIGURATIONS
%----------------------------------------------------------------------------------------                               

\documentclass[fontsize=11pt, % Document font size
                             paper=a4, % Document paper type
                             twoside, % Shifts odd pages to the left for easier reading when printed, can be changed to oneside
                             captions=tableheading,
                             index=totoc,
                             hyperref]{labbook}
 
\usepackage[bottom=10em]{geometry} % Reduces the whitespace at the bottom of the page so more text can fit

\usepackage[english]{babel} % English language
\usepackage{lipsum} % Used for inserting dummy 'Lorem ipsum' text into the template

\usepackage[utf8]{inputenc} % Uses the utf8 input encoding
\usepackage[T1]{fontenc} % Use 8-bit encoding that has 256 glyphs

\usepackage[osf]{mathpazo} % Palatino as the main font
\linespread{1.05}\selectfont % Palatino needs some extra spacing, here 5% extra
\usepackage[scaled=.88]{beramono} % Bera-Monospace
\usepackage[scaled=.86]{berasans} % Bera Sans-Serif

\usepackage{booktabs,array} % Packages for tables

\usepackage{amsmath} % For typesetting math
\usepackage{graphicx} % Required for including images
\usepackage{etoolbox}
\usepackage[norule]{footmisc} % Removes the horizontal rule from footnotes
\usepackage{lastpage} % Counts the number of pages of the document
\usepackage{multirow}

\usepackage[dvipsnames]{xcolor}  % Allows the definition of hex colors
\definecolor{titleblue}{rgb}{0.16,0.24,0.64} % Custom color for the title on the title page
\definecolor{linkcolor}{rgb}{0,0,0.42} % Custom color for links - dark blue at the moment

\addtokomafont{title}{\Huge\color{titleblue}} % Titles in custom blue color
\addtokomafont{chapter}{\color{OliveGreen}} % Lab dates in olive green
\addtokomafont{section}{\color{Sepia}} % Sections in sepia
\addtokomafont{pagehead}{\normalfont\sffamily\color{gray}} % Header text in gray and sans serif
\addtokomafont{caption}{\footnotesize\itshape} % Small italic font size for captions
\addtokomafont{captionlabel}{\upshape\bfseries} % Bold for caption labels
\addtokomafont{descriptionlabel}{\rmfamily}
\setcapwidth[r]{10cm} % Right align caption text
\setkomafont{footnote}{\sffamily} % Footnotes in sans serif

\deffootnote[4cm]{4cm}{1em}{\textsuperscript{\thefootnotemark}} % Indent footnotes to line up with text

\DeclareFixedFont{\textcap}{T1}{phv}{bx}{n}{1.5cm} % Font for main title: Helvetica 1.5 cm
\DeclareFixedFont{\textaut}{T1}{phv}{bx}{n}{0.8cm} % Font for author name: Helvetica 0.8 cm

\usepackage[nouppercase,headsepline]{scrpage2} % Provides headers and footers configuration
\pagestyle{scrheadings} % Print the headers and footers on all pages
\clearscrheadfoot % Clean old definitions if they exist

\automark[chapter]{chapter}
\ohead{\headmark} % Prints outer header

\setlength{\headheight}{25pt} % Makes the header take up a bit of extra space for aesthetics
\setheadsepline{.4pt} % Creates a thin rule under the header
\addtokomafont{headsepline}{\color{lightgray}} % Colors the rule under the header light gray

\ofoot[\normalfont\normalcolor{\thepage\ |\  \pageref{LastPage}}]{\normalfont\normalcolor{\thepage\ |\  \pageref{LastPage}}} % Creates an outer footer of: "current page | total pages"

% These lines make it so each new lab day directly follows the previous one i.e. does not start on a new page - comment them out to separate lab days on new pages
\makeatletter
\patchcmd{\addchap}{\if@openright\cleardoublepage\else\clearpage\fi}{\par}{}{}
\makeatother
\renewcommand*{\chapterpagestyle}{scrheadings}

% These lines make it so every figure and equation in the document is numbered consecutively rather than restarting at 1 for each lab day - comment them out to remove this behavior
\usepackage{chngcntr}
\counterwithout{figure}{labday}
\counterwithout{equation}{labday}

% Hyperlink configuration
\usepackage[
    pdfauthor={Yiqing Hua}, % Your name for the author field in the PDF
    pdftitle={Laboratory Journal}, % PDF title
    pdfsubject={}, % PDF subject
    bookmarksopen=true,
    linktocpage=true,
    urlcolor=linkcolor, % Color of URLs
    citecolor=linkcolor, % Color of citations
    linkcolor=linkcolor, % Color of links to other pages/figures
    backref=page,
    pdfpagelabels=true,
    plainpages=false,
    colorlinks=true, % Turn off all coloring by changing this to false
    bookmarks=true,
    pdfview=FitB]{hyperref}

\usepackage[stretch=10]{microtype} % Slightly tweak font spacing for aesthetics

%\setlength\parindent{0pt} % Uncomment to remove all indentation from paragraphs

%----------------------------------------------------------------------------------------
%	DEFINITION OF EXPERIMENTS
%----------------------------------------------------------------------------------------

% Template: \newexperiment{<abbrev>}[<short form>]{<long form>}
% <abbrev> is the reference to use later in the .tex file in \experiment{}, the <short form> is only used in the table of contents and running title - it is optional, <long form> is what is printed in the lab book itself

\newexperiment{DSE}[dse]{Experiments with DSE}
\newexperiment{Agent}[agent]{Experiments with Agent}
\newexperiment{Target}[target]{Experiments with Target}

\newexperiment{hyper}[dse-sub1]{Hyperparameter Tuning}
\newexperiment{wordvec}[dse-sub2]{Train with different Wordvector}



\newexperiment{original}[target-sub1]{Using the Original RNN Structure}
\newexperiment{duplicate}[target-sub2]{Merge the Input Labels}
\newexperiment{multitask}[target-sub3]{Multitasking}

%----------------------------------------------------------------------------------------

\begin{document}
\printindex
%\tableofcontents % Table of contents
%\newpage % Start lab look on a new page

\begin{addmargin}[4cm]{0cm} % Makes the text width much shorter for a compact look

\pagestyle{scrheadings} % Begin using headers


%----------------------------------------------------------------------------------------
%	LAB BOOK CONTENTS
%----------------------------------------------------------------------------------------

\labday{What We've Got}
\begin{itemize}
	\item Pretty good DSE extraction results. \cite{ozanRNN}.
	\item Better than \cite{bishanJoint}'s Agent extraction results.
	\item Strange and very bad Target extraction results.
\end{itemize}


%-----------------------------------------

\labday{DSE}
We've tried OCLASS weight, drop out and used different word vector settings and we have similar(slightly better?) results than that in \cite{ozanRNN}.

%-----------------------------------------

\labday{Agent}
The agent extraction is great! We have better results on agent extraction than that in \cite{bishanJoint}.

%-----------------------------------------

\labday{Target}

\experiment{original}
We can get the best results from training with the original RNN by tuning the OCLASS weight fairly low, but in this case, the results on validation set can be better than the training set.\\
\begin{table}
\label{tab:original_results}
\raggedleft
\begin{tabular}{c|c|c|c|c|}
\toprule
&\multicolumn{2}{|c|}{OCLASS WEIGHT = 0.09} & \multicolumn{2}{|c|}{OCLASS WEIGHT = 0.25}\\
\hline
  & Prop     & Bin     & Prop     & Bin\\
\hline
P & 0.274956 &0.324164 & 0.375136 & 0.411554 \\
R & 0.702289 &0.808863 & 0.324453 & 0.467323\\
F1&  0.39519 &0.462838 & 0.347958 & 0.437669\\
\bottomrule
\end{tabular}
\caption{Target Extraction using RNN}
\end{table}
Example sentence when Oclass weight is low shows that the mistake usually happens when the RNN thinks a whole span is a target but actually none of the part is included in the target. Sometimes the span do seem look like a named entity. The original results are in oclassLOWres.out. \\ 
All the parameter settings and the output format are in README. Also the full results when Oclass weight is high are in oclassHIGHres.out. 

\experiment{duplicate}

We tried to combine all the labels into one 7-dimensional vector. So the original output is the 3-dimensional vector indicating the probability of the label of a word being B, I or O. Now is the 7-dimensional vector indicating the probability being None, Begining of target/agent/dse or part of them.\\
Since there are overlapping of the tags in the MPQA database, we tried two different ways to generate the input we want. These and the results can be seen in Louis's report. But in general, the target extration is worse than the simpler way we tried above.

\experiment{multitask}

We also tried to do multitasking. The network structure is shown below. But the result is very bad. The good thing about this multitasking network is that we can assign different oclass weight to the three labels, but once these weights are assigned. My guess is there should be some extra tuning with the loss function regarding different label classes but every settings I tried just make things worse. 
\begin{figure}[h!]
\raggedleft
\includegraphics[scale=0.5,keepaspectratio=true]{multitaskRNN.png}
\caption{Multitask RNN}
\label{fig:multi}
\end{figure}
Here's an example of the multitasking results. And the original one is in multi.out
\begin{table}
\label{tab:original_results}
\raggedleft
\begin{tabular}{c|c|c|c|c|c|c}
\toprule
&\multicolumn{2}{|c|}{Target Extraction} & \multicolumn{2}{|c|}{Agent Extraction} & \multicolumn{2}{|c|}{DSE Extraction} \\
\hline
& Prop     & Bin     & Prop     & Bin & Prop & Bin\\
\hline
P & 0.242472 &0.242902 & 0.538143 & 0.54434 &  0.552941  & 0.552941\\
R & 0.311916 &0.727395 & 0.424234 & 0.535906 & 0.0304725 & 0.0342566\\
F1& 0.272845 &0.364189 & 0.474447 & 0.54009 & 0.0577618 & 0.0645161 \\
\bottomrule
\end{tabular}
\caption{Multitasking Results}
\end{table}

\end{addmargin}

%----------------------------------------------------------------------------------------
%	BIBLIOGRAPHY
%----------------------------------------------------------------------------------------

\begin{thebibliography}{9}


\bibitem{ozanRNN}
Ozan Irsoy and Claire Cardie,
Opinion Mining with Deep Recurrent Neural Networks,
In \emph{Proceedings of the Conference on Empirical Methods in Natural Language Processing},
720--728,
2014,
Doha, Qatar,
http://aclweb.org/anthology/D14-1080

\bibitem{bishanJoint}
Bishan Yang and Claire Cardie,
Joint inference for fine-grained opinion extraction,
2013,
In \emph{Proceedings of ACL}

\end{thebibliography}

%----------------------------------------------------------------------------------------

\end{document}
